%%%%%%%%%%%%%%%%%%%%%%%%%%%%%%%%%%%%%%%%%
% Medium Length Professional CV
% LaTeX Template
% Version 2.0 (8/5/13)
%
% This template has been downloaded from:
% http://www.LaTeXTemplates.com
%
% Original author:
% Trey Hunner (http://www.treyhunner.com/)
%
% Important note:
% This template requires the resume.cls file to be in the same directory as the
% .tex file. The resume.cls file provides the resume style used for structuring the
% document.
%
%%%%%%%%%%%%%%%%%%%%%%%%%%%%%%%%%%%%%%%%%

%----------------------------------------------------------------------------------------
%	PACKAGES AND OTHER DOCUMENT CONFIGURATIONS
%----------------------------------------------------------------------------------------

\documentclass{resume} % Use the custom resume.cls style
\usepackage[dvipsnames]{xcolor}
\usepackage{hyperref}

\usepackage[left=0.75in,top=0.6in,right=0.75in,bottom=0.1in]{geometry} % Document margins
\newcommand{\tab}[1]{\hspace{.2667\textwidth}\rlap{#1}}
\newcommand{\itab}[1]{\hspace{0em}\rlap{#1}}



\name{Amir Shamaei} % Your name
% \address{} % Your address

%\address{123 Pleasant Lane \\ City, State 12345} % Your secondary addess (optional)
% \address{GitHub: amirshamaei \\ ResearchGate: amir-shamaei \\ LinkedIn: amshamaei}

\address{ +13682992121 ,  shamaie.amir@gmail.com} % Your phone number and email
\address{Canada,  \url{amirshamaei.github.io}}

\renewenvironment{rSection}[1]{
\sectionskip
\textcolor{RoyalPurple}{\MakeUppercase{#1}}
\sectionlineskip
\hrule
\begin{list}{}{
\setlength{\leftmargin}{1.5em}
}
\item[]
}{
\end{list}
}



\begin{document}
\begin{rSection}{professional summary}
% for teaching
% Biomedical engineer with a Ph.D. and expertise in developing artificial intelligence and software solutions for medical imaging. Currently engaged in teaching roles at the University of Calgary and the Southern Alberta Institute of Technology (SAIT), Canada, where I co-instruct advanced courses in image processing and machine learning, alongside courses in operating systems and software security. Seeking to leverage my strong research background, programming skills, and comprehensive teaching and development experience to advance the field of biomedical imaging. Passionate about applying machine learning techniques to improve the analysis and interpretation of complex medical imaging data, ultimately enhancing patient care.
% # general
An accomplished biomedical researcher with over 8 years of experience in developing cutting-edge deep learning solutions for medical data analysis, reconstruction, and processing. With a Ph.D. in biomedical engineering and extensive research experience, Amir has demonstrated expertise in applying advanced machine learning, deep learning, and signal processing techniques to solve complex problems in medical imaging. Amir's research focuses on pushing the boundaries of AI-driven medical imaging to improve patient outcomes and advance the field of biomedical engineering.
\end{rSection}
%----------------------------------------------------------------------------------------


%----------------------------------------------------------------------------------------
%	WORK EXPERIENCE SECTION
%----------------------------------------------------------------------------------------

\begin{rSection}{EXPERIENCE}

\begin{rSubsection}{Senior DSP/ML Engineer}{2024 - Present}{Synex Medical, Toronto, Canada}{}
    \item Developed advanced signal processing techniques for solving inverse problems in magnetic resonance data, including denoising, filtering, MRI signal reconstruction, and preprocessing. These methods significantly improved signal quality and accuracy in low SNR scenarios, leading to more reliable data interpretation.
    \item Designed and implemented an end-to-end machine learning pipeline with CI/CD integration for model training, tracking, and selection. Leveraged techniques like cross-validation, hyperparameter tuning, and Bayesian optimization to refine models for precise quantification of magnetic resonance signals. Collaborated with software engineering to seamlessly integrate DSP/ML solutions into production systems.
\end{rSubsection}


\begin{rSubsection}{Instructor}{2023 - 2024}{University of Calgary and the Southern Alberta Institute of Technology (SAIT), Canada}{}
    \item Advanced Topics in Image Analysis and Machine Learning  (co-instructor)
    \item Operating Systems
    \item Software Security
\end{rSubsection}

\begin{rSubsection}{Postdoctoral Fellow}{2023 - 2024}{Advanced Imaging and Artificial Intelligence Lab, University of Calgary, Canada}{}
\item Developed an innovative deep learning-based MRI reconstruction framework that integrates a deep registration model and a transformer-based enhancement network,
\item Incorporated prior subject-specific information using deep learning techniques, enabling 4 times faster MRI acquisitions while maintaining high diagnostic quality,
% \item Demonstrated improved accuracy and volumetric agreement with reference segmentations in downstream brain segmentation tasks
\item Developing MRIntelligence, a software application aimed at integrating advanced MRI reconstruction and analysis techniques into clinical workflows.
\end{rSubsection}
\begin{rSubsection}{Marie Skłodowska-Curie Fellow}{2019 - 2023}{Czech Academy of Sciences, Czech Republic}{}
\item Led the research and development of innovative techniques for magnetic resonance spectroscopic imaging analysis, utilizing deep learning algorithms to reduce processing time by 6 orders of magnitude,
% \item Lead research applying deep learning for medical image analysis and reconstruction
\item Designed open-source solutions for magnetic resonance signal processing using Python and Java,
\item Participated in the development of jMRUI software package for magnetic resonance signal analysis, utilized by more than 4,550 researchers worldwide,
\clearpage
\item Ensured compliance with regulatory standards and quality assurance protocols, contributing to the CE/FDA approval of the developed software,
\item Published 10 papers in top conferences and journals.
\end{rSubsection}

% \begin{rSubsection}{Biomedical Software Developer}{2019 - 2023}{Institute of Scientific Instruments of the Czech Academy of Sciences, Czech Republic}{}

% % \item Collaborate closely with a multidisciplinary team of engineers and clinicians to ensure the development of user-friendly and clinically relevant solutions.
% \end{rSubsection}

\begin{rSubsection}{Visiting Researcher}{Summer 2022}{Tesla Dynamic Coils, Netherlands}{}
\item Developed an open-source software package for efficient water signal removal in MR spectroscopic imaging, reducing the computational time by 98%. 
\end{rSubsection}

\begin{rSubsection}{Visiting Researcher}{Winter 2020}{Swiss Institute for Translational and Entrepreneurial Medicine}{}
% \item Developed an AI solution for automating the quantification and processing of magnetic resonance spectroscopy data, which involved creating machine learning models to analyze spectral data from MR scanners
\item Identified potential intrinsic biases in deep learning methods for the quantification of MR spectra.
\end{rSubsection}

\begin{rSubsection}{Visiting Researcher}{Winter 2020}{The University of Bern – Departments of Radiology, Neuroradiology and Nuclear Medicine}{}
\item Participated in the development of FiTAID software package for analyzing multi-dimensional magnetic resonance spectroscopy data, and integrating it into jMRUI software package.
\end{rSubsection}


% \begin{rSubsection}{Teaching Assistant}{2016 - 2017}{Amirkabir University of Technology }{}
% \item Principles of Bio-Electronics, taught course materials and moderated hands-on experience sessions
% \end{rSubsection}




\end{rSection}
%	EDUCATION SECTION
%----------------------------------------------------------------------------------------

\begin{rSection}{Education}


\begin{rSubsection}{ Brno University of Technology} {2019 - 2023} {Ph.D. in Biomedical Engineering}{}
\item Dissertation: Development of new artificial intelligence functionalities for magnetic resonance spectroscopic signals processing
\item Supervisor: Dr. Radovan Jiřík 
\item Co-Supervisors: Dr. Zenon Starčuk, Dr. Jana Starčuková
\end{rSubsection}

\begin{rSubsection}{ Amirkabir University of Technology} {2015 - 2018} {M.Sc. in Biomedical Engineering} {}{}
\item Dissertation: Design and fabrication of a Lab-on-chip device for assessment of cancer cells viability during electrical stimulation
\item Supervisor: Dr. Saviz
\item Co-Supervisors: Dr. Solouk, Dr. Abdullahad
\end{rSubsection}

\begin{rSubsection}{ Zanjan University} {2010 - 2015}{B.Sc. in Electrical Engineering}{}
\item Dissertation: Wireless System Network Simulator Software
\item Supervisor: Dr. Babazadeh
\end{rSubsection}

%Minor in Linguistics \smallskip \\
%Member of Eta Kappa Nu \\
%Member of Upsilon Pi Epsilon \\


\end{rSection}


%	EXAMPLE SECTION
%----------------------------------------------------------------------------------------

\begin{rSection}{Achievements and Awards} \itemsep -2pt
\begin{rSubsection}{}{}{}{}
\item  {First Place, ISMRM MRS Challenge, Boston MA - AI-based Brain Metabolite Fitting}\hfill {\em Fall 2024}
\item  {Lab2Market Validate Program: Offered \$15,000 grant (declined)}\hfill {\em Fall 2024} 
\item  {Awarded \$10,000 AWS cloud credit for research on decreasing MRI acquisition time using deep learning}\hfill {\em Fall 2023} 
\item {Awarded at the annual ISMRM 2023 MR Spectroscopy Study Group Business Meeting}\hfill {\em Spring 2023} 
\item {Ranked 1st in the student competition conference at the Brno University of Technology}\hfill {\em Spring 2020} 
\item {Awarded Thermo Fisher Scientific Corporation’s 2020 prize}\hfill {\em Spring 2020}
\item {Ranked 272\textsuperscript{nd} out of 30,000 participants in the Iranian university entrance exam}\hfill {\em Summer 2015} 
\clearpage
\item{Received funding from the European Union's Horizon 2020 research and innovation program under the Marie Sklodowska-Curie grant agreement No 813120}\hfill {\em 2019-2022} 
\item {Received a full scholarship from Amirkabir University of Technology}\hfill {\em 2015-2018}
% \item {Received a full scholarship from Zanjan University}\hfill {\em 2010-2014}
\end{rSubsection}
\end{rSection}


%----------------------------------------------------------------------------------------
%	Peer-Reviewed Publications
%----------------------------------------------------------------------------------------

\begin{rSection}{ Peer-Reviewed Publications }{}
\begin{rSubsection}{}{}{}{}

\item \textbf{Shamaei, A}, Saviz, M., Solouk, A. et al. An In Vitro Electric Field Exposure Device with Real-Time Cell Impedance Sensing. Iran J Sci Technol Trans Sci 44, 575–585 (2020). doi.org/10.1007/s40995-020-00861-z, Impact factor: 2.4
\item \textbf{Shamaei, A}, Starcukova, J, Pavlova, I, Starcuk, Z. Model-informed unsupervised deep learning approaches to frequency and phase correction of MRS signals. Magn Reson Med. 2023; 89: 1221– 1236. doi:10.1002/mrm.29498,  Impact factor: 3.3

\item Clarke, W, Mikkelsen, M, Oeltzschner, G, Bell T.K.,\textbf{ Shamaei, A}, Soher, B.J., Emir, U, Wilson, W. A standard data format for magnetic resonance spectroscopy. Magnetic Resonance in Medicine 2022; 1- 13. doi:10.1002/mrm.29418, Impact factor: 3.3

\item Rizzo, R, Dziadosz, M, Kyathanahally, SP, \textbf{Shamaei, A}, Kreis, R. Quantification of MR spectra by deep learning in an idealized setting: Investigation of forms of input, network architectures, optimization by ensembles of networks, and training bias. Magn Reson Med. 2022; 1- 21. doi:10.1002/mrm.29561, Impact factor: 3.3

\item \textbf{Shamaei, A}, Starcukova, J, Starcuk, Z. Physics-informed Deep Learning Approach to Quantification of Human Brain Metabolites from Magnetic Resonance Spectroscopy Data. Computers in Biology and Medicine. 2023; 158: 106837. doi:10.1016/j.compbiomed.2023.106837.  , Impact factor: 7.7

\item \textbf{Shamaei, A}, Starcukova J, Rizzo R, Starcuk Z. Water removal in MR spectroscopic imaging with Casorati singular value decomposition. Magn Reson Med. 2024; 91(4): 1694-1706. doi: 10.1002/mrm.29959, Impact factor: 3.3

\item \href{https://openreview.net/forum?id=vu4LsiSpf7&referrer=%5Bthe%20profile%20of%20Abbas%20Omidi%5D(%2Fprofile%3Fid%3D~Abbas_Omidi1)} {\textbf Unsupervised Domain Adaptation of Brain MRI Skull Stripping Trained on Adult Data to Newborns: Combining Synthetic Data with Domain Invariant Features, Medical Imaging with Deep Learning (MIDL), 2024}

\item Dias, G., Berto, R. P., Oliveira, M., Ueda, L., Dertkigil, S., Costa, P. D. P., \textbf{Shamaei, A.}, Bugler, H., Souza, R., Harris, A., Rittner, L. Spectro-ViT: A vision transformer model for GABA-edited MEGA-PRESS reconstruction using spectrograms. Magnetic Resonance Imaging, 113, 110219. doi: https://doi.org/10.1016/j.mri.2024.110219, Impact factor: 3.3
\end{rSubsection}
\end{rSection}

%----------------------------------------------------------------------------------------
%	CONFERENCE PRESENTATIONS
%----------------------------------------------------------------------------------------

\begin{rSection}{CONFERENCE PRESENTATIONS}
\begin{rSubsection}{}{}{}{}

\item \textbf{Shamaei, A}, Niess, E, Strasser, B, , Hingerl, and Bogner, W, Motyka, S. Deep Learning Framework for Quantifying High-Resolution MRSI Data of the Human Brain at 7T, Poster presentation delivered in person at ISMRM 2024, May, 2024.

\item Motyka, S, Niess, E, Strasser, B, \textbf{Shamaei, A}, Hingerl, L, Weiser, P,  Niess, F, and Bogner, W, Towards Prediction of Motion Affected Spectra for MRSI at 7T, Poster presentation delivered in person at ISMRM 2024, May, 2024.

\item \textbf{Shamaei, A}, Starčuková J. and Starcuk Jr. Z., EigenMRS: A Computationally Cheap Data-Driven Approach to MR Spectroscopic Imaging Denoising, Poster presentation delivered in person at ISMRM 2023, June, 2023.
\item \textbf{Shamaei, A}, Starčuková J., Jedrek Burakiewicz, and Starcuk Jr. Z., Water Removal In MR Spectroscopic Imaging with Casorati Singular Value Decomposition, Poster presentation delivered in person at ISMRM 2023, June, 2023.
\item \textbf{Shamaei, A}, Rizzo, R., Physics-Informed Deep Learning Approach to Quantifying Magnetic Resonance Spectroscopy Data with Simultaneous Uncertainty Estimation, Power pitch presentation delivered in person at ISMRM 2023, June, 2023.
\item \textbf{Shamaei, A}, Starčuková J., Radim Kořínek, and Starcuk Jr. Z., Magnetic Resonance Spectroscopic Imaging Data Denoising by Manifold Learning: An Unsupervised Deep Learning Approach, Poster presentation delivered in person at ISMRMB 2022, May, 2022.
\item \textbf{Shamaei, A},, Starčuková J. and Starcuk Jr. Z., Frequency and Phase Shift Correction of MR Spectra Using Deep Learning in Time Domain, Poster presentation delivered virtually at ESMRMB 2021, October, 2021.
Clarke, W., Bell, T., Emir, U., Mikkelsen, M., Oeltzschner, G., Rowland, B., \textbf{Shamaei, A}, Soher, B, Tapper, S., and Wilson, M, NIfTI MRS: A standard format for spectroscopic data, Poster presentation delivered in person at ISMRMB 2021, May, 2021.
\item \textbf{Shamaei, A}, Starčuková J. and Starcuk Jr. Z., A Wavelet Scattering Convolution Network for Magnetic Resonance Spectroscopy Signal Quantitation, Oral Presentation delivered virtually at the 14th International Joint Conference on Biomedical Engineering Systems and Technologies, Vienna, February, 2021, 10.5220/0010318502680275.
\item \textbf{Shamaei, A}, Jiřík R. and Starčuková J., Deep Learning For Magnetic Resonance Spectroscopy Quantification: A Time-Frequency Analysis Approach, Presented in The 26th Conference Student EEICT, Brno University of Technology, March, 2020,  WOS:000598376500032.
\item \textbf{Shamaei, A}, and Saviz, M. Voltage Transfer Functions for In-Vitro Cell Stimulation: A Computational Study, Poster presentation delivered at IEEE Conference on Biomedical Engineering, Amirkabir University of Technology, October, 2017, 10.1109/ICBME.2017.8430245.
\end{rSubsection}
\end{rSection}

%----------------------------------------------------------------------------------------
%	CERTIFICATIONS
%----------------------------------------------------------------------------------------

\begin{rSection}{ CERTIFICATIONS} \itemsep -2pt
\begin{rSubsection}{}{}{}{}
\item {qMRI workshop, Max Planck Institute for Human Cognitive and Brain Sciences, 2023}
\item {Cambridge Ellis Machine Learning Summer School, University of Cambridge, 2022}
\item {INSPiRATION Workshop (Business case game for developing medical devices), Icometrix, Leuven, Belgium, 2022}

\item {Simultaneous PET-MR workshop,  The University of Manchester, 2022}
\item {Workshop on Open-Access and Ethics,  The University of Manchester, 2022}
\item {Soft skills workshop on communication, EPFL, 2021}

\item {Deep Layers workshop, Czech Academy of Science, 2021}
\item {Machine learning: from basics to big data and deep learning in MRI/MRS, KU Leuven, 2021}
\item {jMRUI Software Package workshop, Czech Academy of Science, 2020}

\item {Introduction to clinical MRS, MRS quantitation methods, University of Bern, 2020}
 {Biomedical Signal Processing and Programming (SignalPlant) Workshop, Czech Academy of Science, 2019}
\item {NeuroImaging - Mapping the function and structure of the brain, CEITEC Masaryk University, 2019}
\item {Multiphysics Analysis of bio-model Using COMSOL workshop, 2017}
\end{rSubsection}
\end{rSection}

%----------------------------------------------------------------------------------------
%	OPEN-SOURCE PROJECTS
%----------------------------------------------------------------------------------------

\begin{rSection}{ OPEN-SOURCE PROJECTS} \itemsep -2pt
\begin{rSubsection}{}{}{}{}
\item {\href{https://github.com/amirshamaei/Deep-MRS-Quantification}{Model-Constrained Deep Learning Approach to the Quantification of MR Spectroscopy Data} } 
\item {\href{https://github.com/amirshamaei/DeepFPC}{Model-Informed unsupervised DL Approach to Frequency and Phase Correction of MRS Signals}}
\item {\href{https://github.com/amirshamaei/BrukerEyes}{BrukerEyes: a cross-platform open-source package for Bruker Datasets management}
}
\item {\href{https://github.com/isi-nmr/IDV-Interrelated-MRS-Datasets-Viewer}{A Cross-Platform Open-Source tool for Interrelated MRS(NifTi) Datasets visualization}}
\item {\href{https://github.com/amirshamaei/Frequency-and-Phase-Correction-of-MRS-signals-Using-Cross-Correlation}{Frequency and Phase Correction of MRS signals Using Cross-Correlation}}
\item {\href{https://github.com/isi-nmr/bruker2nii}{A JAVA package for converting Bruker datasets, including MRS and MRSI, to Nifti format}}
\item {\href{https://github.com/amirshamaei/HLSVDPro4J}{HLSVDPro JAVA implementation}}
\item {\href{https://github.com/amirshamaei/NIfTI-MRS}{A JAVA package providing Reading/Writing interface for NIfTi MRS format}}
\item {\href{https://github.com/amirshamaei/niftijio}{A Java library for reading and writing NIfTI (1&2) image volumes}
\end{rSubsection}
\end{rSection}}
%----------------------------------------------------------------------------------------
%	SKILLS
%----------------------------------------------------------------------------------------
\clearpage
\begin{rSection}{Skills}

\begin{tabular}{ @{} >{\bfseries}l @{\hspace{6ex}} l }
Programming Languages &  Java, Python, C(++) \\
Software & MatLab, jMRUI, 3D Slicer, LC-MODEL, FiTAID, Image-J, FreeSurfer
 \\
Framework & Pytorch, JAX, FSL(FMRIB), ITK, MONAI, TorchIO, scikit-image, \\ & SimpleITK, pydicom, and NiBabel \\
\end{tabular}

\end{rSection}

%----------------------------------------------------------------------------------------
%	Languages
%----------------------------------------------------------------------------------------

\begin{rSection}{Languages}

Persian (native), Azeri (native), English (Professional working proficiency), Turkish (Elementary proficiency), Czech (Elementary proficiency), Arabic (Elementary proficiency)

\end{rSection}

\end{document}
